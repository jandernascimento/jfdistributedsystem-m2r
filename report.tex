\documentclass[a4paper]{article}

\usepackage{listings}
\usepackage{color}

\usepackage[top=2.0cm, bottom=2.0cm, left=2.0cm, right=2.0cm]{geometry}

\lstset{numbersep=5pt,numbers=left,numberstyle=\footnotesize,title=\lstname,basicstyle=\footnotesize,showspaces=false,breaklines=true}

\begin{document}

\title{Transactional storage for geo-replicated systems \\ overview}

\author{Jander Nascimento,
\and Liviu Varga}

\maketitle

%Begin Liviu section

\section{Introduction}
This paper will give a short review of Walter, a key-value store that supports transactions and replicates data across remote sites. Those goals are reached by using preferred sites and counting sets, this protocol is named Parallel Snapshot Isolation (PSI). 

PSI was coined after Snapshop Isolation(SI), an approach already largely used by by comercial databases (e.g. Oracle). 

SI provides a strong guarantee within a site; Which leaves the lack for the multiple sites store. Thus, PSI provides casual ordering and precludes write-write conflicts across sites.

The need of such a system is because web applications are getting more popular and are deployed over many data centers or sites around the world to provide better locality, availability and distance tolerance which involves geo-replication. 

The benefit of transaction is to group writes in an atomic unit so that failures do not leave behind partial writes and concurrent access by other services are not intermingled(isolated). Without them there is a significant burden which have to be dealt by application developers. 

The key features of Walter are:

\begin{itemize}
\item \textit{Asynchronous replication across sites.} Replication is done lazily in the background to reduce latency
\item \textit{Efficient update-anywhere for certain objects.} Counting sets can be updated efficiently anywhere, while other objects can be updated efficiently at their preferred site
\item \textit{Freedom from conflict-resolution logic}. which is complex to developers
\item \textit{Strong isolation within each site.} Provided by PSI.
\end{itemize}
          
\section{Parallel Snapshot Isolation}
Snapshot isolation ensures that, \textbf{first} transactions are read from a snapshot that reflects a single commit ordering of transactions, and \textbf{secondly} providing certain level of non-assisted conflict resolution, e.g if two concurrent transactions have a write-write conflict, one must be aborted.
 
\textit{Parallel snapshot isolation} extends Snapshot Isolation by allowing different sites to have different commit orderings across multiple sites. PSI's relaxation is acceptable for web applications where each user communicates with one site at a time and there is no need for a global ordering of all actions across all users. 

To avoid write-write conflicts across sites and for implementation of PSI, Walter uses two techniques:
\begin{itemize}
\item \textit{Preferred sites.} Each object is assigned a preferred site which is the site where writes to the object can be committed without checking other sites for write conflicts.
\item \textit{Conflict-free counting set objects.} A cset is similar to a multiset in that it keeps a count for each element, but unlike multisets, the count could be negative. A cset support an operation $add(x)$ to add element $x$, which increments the counter of $x$ in the cset; and an operation $rem(x)$ to remove $x$, which decrements the counter of $x$. Because increment and decrement commute, $add$ and $rem$ also commute, and so operations never conflict. 
\end{itemize}

PSI property specification:

PSI \textsc{property} 1. \textit{(Site Snapshot Read) All operations read the most recent committed version at the transaction's site as of the time when the transaction began.}

PSI \textsc{property} 2. \textit{(No Write-Write Conflicts) The write sets of each pair of committed somewhere-concurrent transactions must be disjoint.}

PSI \textsc{property} 3. \textit{(Commit Causality Across Sites) If a transaction $T_{1}$ commits at a site A before a transaction $T_{2}$ starts at site A, then $T_{1}$ cannot commit after $T_{2}$ at any site.}

%may be add some other info
It is said that the transactions are \textit{somewhere-concurrent} if they are concurrent at $site(T_{1})$ or at $site(T_{2})$.

The anomalies which might appear because of the isolation properties of PSI are \textbf{short fork} and \textbf{long fork}, which are the price paid for replicate the store update buffer asynchronously. Others anomalies such as dirty read, non-repeatable read, lost update or conflicting fork does not happen in this protocol. 
%end Liviu section

%start Jander section

\section{Design and Algorithm}

Each site has a Walter server and several clients. For a given object is responsibility of one, and only one, Walter server to update (perform write) that object, although others clients, located in other sites, can read it.

The whole communication between sites are done with Remote Procedure Calls. %we should adding something else in here.. but what!? not sure yet.

For a matter of organization, the objects can be organized in what is called Containers, which a single site may have several. For instance, the administrator can create one Container per user, so he can move the container across other sites to improve locality. Although, it is not possible to move the object from a given container to another one, since the \textit{oid} of each object is composed with the container id, this move would require updating the \textit{oid}'s. 

To provide a site fault tolerance, Walter relies on a \textit{Configuration Service}. Which is responsible for leasing the role of preferred site to containers of the faulty site. In order to perform such operation the Configuration Service keeps a list of sites which are up and running. 

For keeping track the sequence of operations, Walter uses vector timestamps to synchronize between sites, since the monolic-time concept is expensive and causes unnecessary dependencies(not causal ordering). Each site has a local version number indicating the number of transactions reflected in the snapshot.

There exist two types of transactions: the fast and slow commit. 

The \textit{fast commit} is used for objects which are local to the preferred site and involves two checks: first, checking if the objects are unmodified since the transaction started and second if all the objects are unlocked. 
In other corner, \textit{slow commit} is used for transactions whose object's preferred site is a non-local one. In this case to update is done in two phases: first, the server asks the objects preferred sites if they are unmodified and unlocked and if all of them are "yes" then it proceeds to commit as a fast commit protocol. %review once more

Propagation of committed transactions are done asynchronously to the other sites. When all the transactions are committed at all sites, the transaction is marked as globally thus becomes visible to other sites. 

\section{Evaluation}

They first evaluated the base performance of Walter comparing it with Berkely DB 11gR2. The resulted transaction throughputs of read and write were comparable.

The benchmark consisted in either write or read transactions each accessing one 100 byte object. Experiments were run on Amazon's EC2 cloud platform in four sites across the globe. 

The micro-benchmark for evaluating the throughput of transactions across several sites with read/write operations as well as performing several operations within one transactions showed a linear scalability with the number of sites. Read throughput is bounded by the RPC performance and write throughput is lower compared to the read one because of the lock contention. Having several operations within one transactions increases the throughput compared with every operation being a single transaction. 

As far as the latency concerns, when having fast commits this was mainly influenced by the effects of queuing inside the Walter server and of flushing the commit log to disk resulting a 99.9 percentile latency of 27 ms. When it comes to disaster safe transactions the latency is approximately uniform distributed between $[RTT_{MAX},2*RTT_{MAX}]$ where $RTT_{MAX}$ is the maximum round trip latency between the site which issued the transactions and other sites.

%end Jander section

\section{Conclusion}

The authors have designed a transactional geo-replicated key-value store with the key feature of Parallel Snapshot Isolation which in case of asynchronous replication, avoids conflicts across sites allowing efficient implementation of application, demonstrated by WaltSocial and ReTwis which achieved reasonable performance.  

\end{document}